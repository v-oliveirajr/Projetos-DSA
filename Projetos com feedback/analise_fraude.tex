% Options for packages loaded elsewhere
\PassOptionsToPackage{unicode}{hyperref}
\PassOptionsToPackage{hyphens}{url}
%
\documentclass[
]{article}
\usepackage{lmodern}
\usepackage{amsmath}
\usepackage{ifxetex,ifluatex}
\ifnum 0\ifxetex 1\fi\ifluatex 1\fi=0 % if pdftex
  \usepackage[T1]{fontenc}
  \usepackage[utf8]{inputenc}
  \usepackage{textcomp} % provide euro and other symbols
  \usepackage{amssymb}
\else % if luatex or xetex
  \usepackage{unicode-math}
  \defaultfontfeatures{Scale=MatchLowercase}
  \defaultfontfeatures[\rmfamily]{Ligatures=TeX,Scale=1}
\fi
% Use upquote if available, for straight quotes in verbatim environments
\IfFileExists{upquote.sty}{\usepackage{upquote}}{}
\IfFileExists{microtype.sty}{% use microtype if available
  \usepackage[]{microtype}
  \UseMicrotypeSet[protrusion]{basicmath} % disable protrusion for tt fonts
}{}
\makeatletter
\@ifundefined{KOMAClassName}{% if non-KOMA class
  \IfFileExists{parskip.sty}{%
    \usepackage{parskip}
  }{% else
    \setlength{\parindent}{0pt}
    \setlength{\parskip}{6pt plus 2pt minus 1pt}}
}{% if KOMA class
  \KOMAoptions{parskip=half}}
\makeatother
\usepackage{xcolor}
\IfFileExists{xurl.sty}{\usepackage{xurl}}{} % add URL line breaks if available
\IfFileExists{bookmark.sty}{\usepackage{bookmark}}{\usepackage{hyperref}}
\hypersetup{
  pdftitle={Análise de fraude em cliques para download de aplicativo},
  pdfauthor={Vanderlei Oliveira},
  hidelinks,
  pdfcreator={LaTeX via pandoc}}
\urlstyle{same} % disable monospaced font for URLs
\usepackage[margin=1in]{geometry}
\usepackage{color}
\usepackage{fancyvrb}
\newcommand{\VerbBar}{|}
\newcommand{\VERB}{\Verb[commandchars=\\\{\}]}
\DefineVerbatimEnvironment{Highlighting}{Verbatim}{commandchars=\\\{\}}
% Add ',fontsize=\small' for more characters per line
\usepackage{framed}
\definecolor{shadecolor}{RGB}{248,248,248}
\newenvironment{Shaded}{\begin{snugshade}}{\end{snugshade}}
\newcommand{\AlertTok}[1]{\textcolor[rgb]{0.94,0.16,0.16}{#1}}
\newcommand{\AnnotationTok}[1]{\textcolor[rgb]{0.56,0.35,0.01}{\textbf{\textit{#1}}}}
\newcommand{\AttributeTok}[1]{\textcolor[rgb]{0.77,0.63,0.00}{#1}}
\newcommand{\BaseNTok}[1]{\textcolor[rgb]{0.00,0.00,0.81}{#1}}
\newcommand{\BuiltInTok}[1]{#1}
\newcommand{\CharTok}[1]{\textcolor[rgb]{0.31,0.60,0.02}{#1}}
\newcommand{\CommentTok}[1]{\textcolor[rgb]{0.56,0.35,0.01}{\textit{#1}}}
\newcommand{\CommentVarTok}[1]{\textcolor[rgb]{0.56,0.35,0.01}{\textbf{\textit{#1}}}}
\newcommand{\ConstantTok}[1]{\textcolor[rgb]{0.00,0.00,0.00}{#1}}
\newcommand{\ControlFlowTok}[1]{\textcolor[rgb]{0.13,0.29,0.53}{\textbf{#1}}}
\newcommand{\DataTypeTok}[1]{\textcolor[rgb]{0.13,0.29,0.53}{#1}}
\newcommand{\DecValTok}[1]{\textcolor[rgb]{0.00,0.00,0.81}{#1}}
\newcommand{\DocumentationTok}[1]{\textcolor[rgb]{0.56,0.35,0.01}{\textbf{\textit{#1}}}}
\newcommand{\ErrorTok}[1]{\textcolor[rgb]{0.64,0.00,0.00}{\textbf{#1}}}
\newcommand{\ExtensionTok}[1]{#1}
\newcommand{\FloatTok}[1]{\textcolor[rgb]{0.00,0.00,0.81}{#1}}
\newcommand{\FunctionTok}[1]{\textcolor[rgb]{0.00,0.00,0.00}{#1}}
\newcommand{\ImportTok}[1]{#1}
\newcommand{\InformationTok}[1]{\textcolor[rgb]{0.56,0.35,0.01}{\textbf{\textit{#1}}}}
\newcommand{\KeywordTok}[1]{\textcolor[rgb]{0.13,0.29,0.53}{\textbf{#1}}}
\newcommand{\NormalTok}[1]{#1}
\newcommand{\OperatorTok}[1]{\textcolor[rgb]{0.81,0.36,0.00}{\textbf{#1}}}
\newcommand{\OtherTok}[1]{\textcolor[rgb]{0.56,0.35,0.01}{#1}}
\newcommand{\PreprocessorTok}[1]{\textcolor[rgb]{0.56,0.35,0.01}{\textit{#1}}}
\newcommand{\RegionMarkerTok}[1]{#1}
\newcommand{\SpecialCharTok}[1]{\textcolor[rgb]{0.00,0.00,0.00}{#1}}
\newcommand{\SpecialStringTok}[1]{\textcolor[rgb]{0.31,0.60,0.02}{#1}}
\newcommand{\StringTok}[1]{\textcolor[rgb]{0.31,0.60,0.02}{#1}}
\newcommand{\VariableTok}[1]{\textcolor[rgb]{0.00,0.00,0.00}{#1}}
\newcommand{\VerbatimStringTok}[1]{\textcolor[rgb]{0.31,0.60,0.02}{#1}}
\newcommand{\WarningTok}[1]{\textcolor[rgb]{0.56,0.35,0.01}{\textbf{\textit{#1}}}}
\usepackage{graphicx}
\makeatletter
\def\maxwidth{\ifdim\Gin@nat@width>\linewidth\linewidth\else\Gin@nat@width\fi}
\def\maxheight{\ifdim\Gin@nat@height>\textheight\textheight\else\Gin@nat@height\fi}
\makeatother
% Scale images if necessary, so that they will not overflow the page
% margins by default, and it is still possible to overwrite the defaults
% using explicit options in \includegraphics[width, height, ...]{}
\setkeys{Gin}{width=\maxwidth,height=\maxheight,keepaspectratio}
% Set default figure placement to htbp
\makeatletter
\def\fps@figure{htbp}
\makeatother
\setlength{\emergencystretch}{3em} % prevent overfull lines
\providecommand{\tightlist}{%
  \setlength{\itemsep}{0pt}\setlength{\parskip}{0pt}}
\setcounter{secnumdepth}{-\maxdimen} % remove section numbering
\ifluatex
  \usepackage{selnolig}  % disable illegal ligatures
\fi

\title{Análise de fraude em cliques para download de aplicativo}
\author{Vanderlei Oliveira}
\date{3/27/2021}

\begin{document}
\maketitle

\hypertarget{carregamento-dos-dados}{%
\section{Carregamento dos dados}\label{carregamento-dos-dados}}

\begin{Shaded}
\begin{Highlighting}[]
\FunctionTok{library}\NormalTok{(readr)}
\FunctionTok{library}\NormalTok{(data.table)}
\FunctionTok{library}\NormalTok{(dplyr)}
\end{Highlighting}
\end{Shaded}

\begin{verbatim}
## 
## Attaching package: 'dplyr'
\end{verbatim}

\begin{verbatim}
## The following objects are masked from 'package:data.table':
## 
##     between, first, last
\end{verbatim}

\begin{verbatim}
## The following objects are masked from 'package:stats':
## 
##     filter, lag
\end{verbatim}

\begin{verbatim}
## The following objects are masked from 'package:base':
## 
##     intersect, setdiff, setequal, union
\end{verbatim}

\begin{Shaded}
\begin{Highlighting}[]
\FunctionTok{library}\NormalTok{(ROSE)}
\end{Highlighting}
\end{Shaded}

\begin{verbatim}
## Loaded ROSE 0.0-3
\end{verbatim}

\begin{Shaded}
\begin{Highlighting}[]
\NormalTok{df }\OtherTok{\textless{}{-}} \FunctionTok{fread}\NormalTok{(}\AttributeTok{file =} \StringTok{\textquotesingle{}train\_sample.csv\textquotesingle{}}\NormalTok{,}\AttributeTok{header =}\NormalTok{ T,}\AttributeTok{stringsAsFactors =}\NormalTok{ T,}\AttributeTok{drop =} \DecValTok{1}\NormalTok{, }\AttributeTok{data.table=}\NormalTok{F)}
\FunctionTok{table}\NormalTok{(df}\SpecialCharTok{$}\NormalTok{is\_attributed)}
\end{Highlighting}
\end{Shaded}

\begin{verbatim}
## 
##     0     1 
## 99773   227
\end{verbatim}

\begin{Shaded}
\begin{Highlighting}[]
\FunctionTok{View}\NormalTok{(df)}
\NormalTok{df }\OtherTok{\textless{}{-}}\NormalTok{ df}\SpecialCharTok{\%\textgreater{}\%}
  \FunctionTok{select}\NormalTok{(}\SpecialCharTok{{-}}\NormalTok{attributed\_time)}

\DecValTok{100}\SpecialCharTok{*}\FunctionTok{prop.table}\NormalTok{(}\FunctionTok{table}\NormalTok{(df}\SpecialCharTok{$}\NormalTok{is\_attributed))}
\end{Highlighting}
\end{Shaded}

\begin{verbatim}
## 
##      0      1 
## 99.773  0.227
\end{verbatim}

\hypertarget{a-variuxe1vel-alvo-is_attributed-estuxe1-desbalanceada-iremos-fazer-um-oversampling-para-chegarmos-a-uma-proporuxe7uxe3o-95-05}{%
\section{A variável-alvo is\_attributed está desbalanceada, iremos fazer
um oversampling para chegarmos a uma proporção
95\%-0,5\%}\label{a-variuxe1vel-alvo-is_attributed-estuxe1-desbalanceada-iremos-fazer-um-oversampling-para-chegarmos-a-uma-proporuxe7uxe3o-95-05}}

\hypertarget{transformando-a-coluna-app-para-faixas-de-modelos-de-aplicativos}{%
\section{Transformando a coluna ``app'' para faixas de modelos de
aplicativos}\label{transformando-a-coluna-app-para-faixas-de-modelos-de-aplicativos}}

\hypertarget{esta-coluna-teruxe1-5-faixas-de-modelos}{%
\section{Esta coluna terá 5 faixas de
modelos}\label{esta-coluna-teruxe1-5-faixas-de-modelos}}

\begin{Shaded}
\begin{Highlighting}[]
\ControlFlowTok{for}\NormalTok{ (i }\ControlFlowTok{in} \DecValTok{1}\SpecialCharTok{:}\FunctionTok{nrow}\NormalTok{(bal)) \{}
  \ControlFlowTok{if}\NormalTok{(bal2[i,}\DecValTok{1}\NormalTok{]}\SpecialCharTok{\textless{}=}\DecValTok{20}\NormalTok{)bal[i,}\DecValTok{1}\NormalTok{]}\OtherTok{\textless{}{-}}\StringTok{\textquotesingle{}0{-}20\textquotesingle{}}
  \ControlFlowTok{if}\NormalTok{((bal2[i,}\DecValTok{1}\NormalTok{]}\SpecialCharTok{\textgreater{}}\DecValTok{20}  \SpecialCharTok{\&}\NormalTok{ bal2[i,}\DecValTok{1}\NormalTok{]}\SpecialCharTok{\textless{}=}\DecValTok{40}\NormalTok{)}\SpecialCharTok{==}\ConstantTok{TRUE}\NormalTok{)bal[i,}\DecValTok{1}\NormalTok{]}\OtherTok{\textless{}{-}}\StringTok{\textquotesingle{}20{-}40\textquotesingle{}}
  \ControlFlowTok{if}\NormalTok{((bal2[i,}\DecValTok{1}\NormalTok{]}\SpecialCharTok{\textgreater{}}\DecValTok{40}  \SpecialCharTok{\&}\NormalTok{ bal2[i,}\DecValTok{1}\NormalTok{]}\SpecialCharTok{\textless{}=}\DecValTok{100}\NormalTok{)}\SpecialCharTok{==}\ConstantTok{TRUE}\NormalTok{)bal[i,}\DecValTok{1}\NormalTok{]}\OtherTok{\textless{}{-}}\StringTok{\textquotesingle{}40{-}100\textquotesingle{}}
  \ControlFlowTok{if}\NormalTok{((bal2[i,}\DecValTok{1}\NormalTok{]}\SpecialCharTok{\textgreater{}}\DecValTok{100}  \SpecialCharTok{\&}\NormalTok{ bal2[i,}\DecValTok{1}\NormalTok{]}\SpecialCharTok{\textless{}=}\DecValTok{200}\NormalTok{)}\SpecialCharTok{==}\ConstantTok{TRUE}\NormalTok{)bal[i,}\DecValTok{1}\NormalTok{]}\OtherTok{\textless{}{-}}\StringTok{\textquotesingle{}100{-}200\textquotesingle{}}
  \ControlFlowTok{if}\NormalTok{(bal2[i,}\DecValTok{1}\NormalTok{]}\SpecialCharTok{\textgreater{}}\DecValTok{200}\NormalTok{) bal[i,}\DecValTok{1}\NormalTok{]}\OtherTok{\textless{}{-}}\StringTok{\textquotesingle{}200{-}550\textquotesingle{}}
  
\NormalTok{\}}

\FunctionTok{unique}\NormalTok{(bal}\SpecialCharTok{$}\NormalTok{app)}
\end{Highlighting}
\end{Shaded}

\begin{verbatim}
## [1] "0-20"    "20-40"   "40-100"  "100-200" "200-550"
\end{verbatim}

\begin{Shaded}
\begin{Highlighting}[]
\FunctionTok{table}\NormalTok{(bal}\SpecialCharTok{$}\NormalTok{app)}
\end{Highlighting}
\end{Shaded}

\begin{verbatim}
## 
##    0-20 100-200   20-40 200-550  40-100 
##   91438     605   10487     146    2337
\end{verbatim}

\begin{Shaded}
\begin{Highlighting}[]
\FunctionTok{table}\NormalTok{(bal}\SpecialCharTok{$}\NormalTok{device)}
\end{Highlighting}
\end{Shaded}

\begin{verbatim}
## 
##     0     1     2     4     5     6     7     9    11    16    17    18    20 
##  1688 97578  4383    27     7    22     2     1     1    69     1     1     2 
##    21    25    30    33    36    37    40    49    50    53    56    58    59 
##    56     3    27    29    22     1    22     1    25     1    20     1    12 
##    60    67    74    76    78    79    97   100   102   103   106   109   114 
##    22     2    21     1     2     1    81     1    24     1     1    24     2 
##   116   124   129   154   160   163   167   180   182   188   196   202   203 
##    25     2     1     1     2     1     1    15     2    22     1     1     1 
##   210   211   220   241   268   291   321   329   347   351   362   374   385 
##     1     2     1     1     1     1     1     1     1    11     1     1     1 
##   386   414   420   486   516   549   552   558   579   581   596   607   657 
##     1     1     1     1     1     1     1     2    32    25     1     1     1 
##   828   883   928   957  1042  1080  1162  1318  1422  1482  1728  1839  2120 
##     1     1     1    27     1     1     1     1     1     1     1     1     1 
##  2429  2980  3032  3282  3331  3543  3545  3866  3867 
##     1     1   371     1     1   151     1    93     1
\end{verbatim}

\hypertarget{para-a-coluna-device-que-indica-o-tipo-de-smartphone-em-uso-pelo-usuuxe1rio-predominam-se-os-valores-0-1-e-2}{%
\section{Para a coluna device, que indica o tipo de smartphone em uso
pelo usuário, predominam-se os valores 0, 1 e
2}\label{para-a-coluna-device-que-indica-o-tipo-de-smartphone-em-uso-pelo-usuuxe1rio-predominam-se-os-valores-0-1-e-2}}

\hypertarget{entuxe3o-substituiremos-os-valores-maiores-que-2-para-other}{%
\section{Então Substituiremos os valores maiores que 2 para
``Other''}\label{entuxe3o-substituiremos-os-valores-maiores-que-2-para-other}}

\begin{Shaded}
\begin{Highlighting}[]
\NormalTok{bal }\OtherTok{\textless{}{-}}\NormalTok{ bal }\SpecialCharTok{\%\textgreater{}\%} \FunctionTok{mutate}\NormalTok{(}\AttributeTok{device\_n =}\NormalTok{ bal[,}\DecValTok{2}\NormalTok{])}

\ControlFlowTok{for}\NormalTok{ (i }\ControlFlowTok{in} \DecValTok{1}\SpecialCharTok{:}\FunctionTok{nrow}\NormalTok{(bal))\{}
  \ControlFlowTok{if}\NormalTok{(bal[i,}\DecValTok{2}\NormalTok{]}\SpecialCharTok{\textgreater{}}\DecValTok{3}\NormalTok{)bal[i,}\DecValTok{7}\NormalTok{]}\OtherTok{\textless{}{-}}\StringTok{\textquotesingle{}Other\textquotesingle{}}
\NormalTok{\}}

\FunctionTok{unique}\NormalTok{(bal}\SpecialCharTok{$}\NormalTok{device\_n)}
\end{Highlighting}
\end{Shaded}

\begin{verbatim}
## [1] "1"     "2"     "Other" "0"
\end{verbatim}

\begin{Shaded}
\begin{Highlighting}[]
\FunctionTok{table}\NormalTok{(bal}\SpecialCharTok{$}\NormalTok{device\_n)}
\end{Highlighting}
\end{Shaded}

\begin{verbatim}
## 
##     0     1     2 Other 
##  1688 97578  4383  1364
\end{verbatim}

\begin{Shaded}
\begin{Highlighting}[]
\NormalTok{bal }\OtherTok{\textless{}{-}}\NormalTok{ bal}\SpecialCharTok{\%\textgreater{}\%}\FunctionTok{select}\NormalTok{(}\SpecialCharTok{{-}}\NormalTok{device)}
\FunctionTok{head}\NormalTok{(bal)}
\end{Highlighting}
\end{Shaded}

\begin{verbatim}
##     app os channel          click_time is_attributed device_n
## 1  0-20 13     497 2017-11-07 09:30:38             0        1
## 2 20-40 17     259 2017-11-07 13:40:27             0        1
## 3  0-20 19     212 2017-11-07 18:05:24             0        1
## 4  0-20 13     477 2017-11-07 04:58:08             0        1
## 5  0-20  1     178 2017-11-09 09:00:09             0        1
## 6  0-20 17     115 2017-11-09 01:22:13             0        1
\end{verbatim}

\hypertarget{ainda-resta-a-coluna-os-relacionada-ao-sistema-operacional-dos-smartphones}{%
\section{Ainda resta a coluna ``os'' relacionada ao sistema operacional
dos
smartphones}\label{ainda-resta-a-coluna-os-relacionada-ao-sistema-operacional-dos-smartphones}}

\hypertarget{tambuxe9m-seruxe1-colocada-em-faixas-de-valores-por-conter-muitos-valores-uxfanicos}{%
\section{Também será colocada em faixas de valores por conter muitos
valores
únicos}\label{tambuxe9m-seruxe1-colocada-em-faixas-de-valores-por-conter-muitos-valores-uxfanicos}}

\begin{Shaded}
\begin{Highlighting}[]
\NormalTok{bal2 }\OtherTok{\textless{}{-}}\NormalTok{ bal}
\FunctionTok{table}\NormalTok{(bal}\SpecialCharTok{$}\NormalTok{os)}
\end{Highlighting}
\end{Shaded}

\begin{verbatim}
## 
##     0     1     2     3     4     5     6     7     8     9    10    11    12 
##   758  1209   420  1604   368    91  2570   214  2775  2365  2867   752  1129 
##    13    14    15    16    17    18    19    20    21    22    23    24    25 
## 21769  1364  2494  1741  5377  4919 24793  2416   211  4168  1071   797  2358 
##    26    27    28    29    30    31    32    34    35    36    37    38    39 
##   524  1141   704   297   770   449   966   143   934   449  1628   149    49 
##    40    41    42    43    44    45    46    47    48    49    50    52    53 
##   387  1348   195   233    82     7   159   862   113   334    32    39   759 
##    55    56    57    58    59    60    61    62    63    64    65    66    67 
##    44    56    16   158    67     4    75    16    19    36    51    36     3 
##    68    69    70    71    73    74    76    77    78    79    80    81    83 
##     2     3    45     3    26     2    15    26     2    44     4     4     9 
##    84    85    87    88    90    92    96    97    98    99   100   102   106 
##     1     4     4     1    15     5    10    28    10     1    12     5     1 
##   107   108   109   110   111   112   113   114   116   117   118   127   129 
##     2     2     6     3     2     4     1     1     1     2     2     1     1 
##   132   133   135   137   138   142   151   152   153   155   168   172   174 
##     2     1     1     1     2     1     1     2     1     3     1     1     1 
##   178   184   185   192   193   196   198   199   207   607   748   836   866 
##     5     2     1     1     1     2     3     1     1   382   191     1   160
\end{verbatim}

\begin{Shaded}
\begin{Highlighting}[]
\FunctionTok{length}\NormalTok{(}\FunctionTok{table}\NormalTok{(bal}\SpecialCharTok{$}\NormalTok{os))}
\end{Highlighting}
\end{Shaded}

\begin{verbatim}
## [1] 130
\end{verbatim}

\begin{Shaded}
\begin{Highlighting}[]
\ControlFlowTok{for}\NormalTok{ (i }\ControlFlowTok{in} \DecValTok{1}\SpecialCharTok{:}\FunctionTok{nrow}\NormalTok{(bal)) \{}
  \ControlFlowTok{if}\NormalTok{(bal2[i,}\DecValTok{2}\NormalTok{]}\SpecialCharTok{\textless{}=}\DecValTok{15}\NormalTok{)bal[i,}\DecValTok{2}\NormalTok{]}\OtherTok{\textless{}{-}}\StringTok{\textquotesingle{}0{-}15\textquotesingle{}}
  \ControlFlowTok{if}\NormalTok{((bal2[i,}\DecValTok{2}\NormalTok{]}\SpecialCharTok{\textgreater{}}\DecValTok{15}  \SpecialCharTok{\&}\NormalTok{ bal2[i,}\DecValTok{2}\NormalTok{]}\SpecialCharTok{\textless{}=}\DecValTok{30}\NormalTok{)}\SpecialCharTok{==}\ConstantTok{TRUE}\NormalTok{)bal[i,}\DecValTok{2}\NormalTok{]}\OtherTok{\textless{}{-}}\StringTok{\textquotesingle{}15{-}30\textquotesingle{}}
  \ControlFlowTok{if}\NormalTok{((bal2[i,}\DecValTok{2}\NormalTok{]}\SpecialCharTok{\textgreater{}}\DecValTok{30}  \SpecialCharTok{\&}\NormalTok{ bal2[i,}\DecValTok{2}\NormalTok{]}\SpecialCharTok{\textless{}=}\DecValTok{45}\NormalTok{)}\SpecialCharTok{==}\ConstantTok{TRUE}\NormalTok{)bal[i,}\DecValTok{2}\NormalTok{]}\OtherTok{\textless{}{-}}\StringTok{\textquotesingle{}30{-}45\textquotesingle{}}
  \ControlFlowTok{if}\NormalTok{(bal2[i,}\DecValTok{2}\NormalTok{]}\SpecialCharTok{\textgreater{}}\DecValTok{45}\NormalTok{) bal[i,}\DecValTok{2}\NormalTok{]}\OtherTok{\textless{}{-}}\StringTok{\textquotesingle{}Other\textquotesingle{}}
  
\NormalTok{\}}
\FunctionTok{unique}\NormalTok{(bal}\SpecialCharTok{$}\NormalTok{os)}
\end{Highlighting}
\end{Shaded}

\begin{verbatim}
## [1] "0-15"  "15-30" "Other" "30-45"
\end{verbatim}

\# Por fim, também transformaremos a coluna channel em faixas de valores

\begin{Shaded}
\begin{Highlighting}[]
\FunctionTok{unique}\NormalTok{(bal}\SpecialCharTok{$}\NormalTok{channel)}
\end{Highlighting}
\end{Shaded}

\begin{verbatim}
##   [1] 497 259 212 477 178 115 135 442 364 489 205 125 280 349 265 459 215 101
##  [19] 122 379 386 124 140 107 245 111 134 401 137 145 278 409 153 466 128 481
##  [37] 334 424 406 373 377 435 452 445 439 242 315 237   3 480 116 121 400 376
##  [55] 469 130 371 467 113 211 496 219 478 463 105  21 347 234 236 328 173 244
##  [73] 232 266 258 262 127 319 340 412  19 243 360 417 317 160 110 282  30 421
##  [91] 208 402 213 416 325 326 224 391 448 484  13 118 126 343 253 150 487 430
## [109] 182 453 449 268  17 123  22 456 320 120 330 277 479 420  18 203 486 446
## [127] 450 272 333 210 361 490 483 488 171 138 322 457 404 411   5 332 410 274
## [145] 356 460  24 353 498 174 451 419 341 108 474  15 455 261   4 114 465
\end{verbatim}

\begin{Shaded}
\begin{Highlighting}[]
\FunctionTok{View}\NormalTok{(}\FunctionTok{table}\NormalTok{(bal}\SpecialCharTok{$}\NormalTok{channel))}
\FunctionTok{View}\NormalTok{(bal2)}
\ControlFlowTok{for}\NormalTok{ (i }\ControlFlowTok{in} \DecValTok{1}\SpecialCharTok{:}\FunctionTok{nrow}\NormalTok{(bal)) \{}
  \ControlFlowTok{if}\NormalTok{(bal2[i,}\DecValTok{3}\NormalTok{]}\SpecialCharTok{\textless{}=}\DecValTok{100}\NormalTok{)bal[i,}\DecValTok{3}\NormalTok{]}\OtherTok{\textless{}{-}}\StringTok{\textquotesingle{}0{-}100\textquotesingle{}}
  \ControlFlowTok{if}\NormalTok{((bal2[i,}\DecValTok{3}\NormalTok{]}\SpecialCharTok{\textgreater{}}\DecValTok{100}  \SpecialCharTok{\&}\NormalTok{ bal2[i,}\DecValTok{3}\NormalTok{]}\SpecialCharTok{\textless{}=}\DecValTok{200}\NormalTok{)}\SpecialCharTok{==}\ConstantTok{TRUE}\NormalTok{)bal[i,}\DecValTok{3}\NormalTok{]}\OtherTok{\textless{}{-}}\StringTok{\textquotesingle{}100{-}200\textquotesingle{}}
  \ControlFlowTok{if}\NormalTok{((bal2[i,}\DecValTok{3}\NormalTok{]}\SpecialCharTok{\textgreater{}}\DecValTok{200}  \SpecialCharTok{\&}\NormalTok{ bal2[i,}\DecValTok{3}\NormalTok{]}\SpecialCharTok{\textless{}=}\DecValTok{300}\NormalTok{)}\SpecialCharTok{==}\ConstantTok{TRUE}\NormalTok{)bal[i,}\DecValTok{3}\NormalTok{]}\OtherTok{\textless{}{-}}\StringTok{\textquotesingle{}200{-}300\textquotesingle{}}
  \ControlFlowTok{if}\NormalTok{((bal2[i,}\DecValTok{3}\NormalTok{]}\SpecialCharTok{\textgreater{}}\DecValTok{300}  \SpecialCharTok{\&}\NormalTok{ bal2[i,}\DecValTok{3}\NormalTok{]}\SpecialCharTok{\textless{}=}\DecValTok{400}\NormalTok{)}\SpecialCharTok{==}\ConstantTok{TRUE}\NormalTok{)bal[i,}\DecValTok{3}\NormalTok{]}\OtherTok{\textless{}{-}}\StringTok{\textquotesingle{}300{-}400\textquotesingle{}}
  \ControlFlowTok{if}\NormalTok{(bal2[i,}\DecValTok{3}\NormalTok{]}\SpecialCharTok{\textgreater{}}\DecValTok{400}\NormalTok{) bal[i,}\DecValTok{3}\NormalTok{]}\OtherTok{\textless{}{-}}\StringTok{\textquotesingle{}400{-}500\textquotesingle{}}
  
\NormalTok{\}}
\FunctionTok{unique}\NormalTok{(bal}\SpecialCharTok{$}\NormalTok{channel)}
\end{Highlighting}
\end{Shaded}

\begin{verbatim}
## [1] "400-500" "200-300" "100-200" "300-400" "0-100"
\end{verbatim}

\hypertarget{anuxe1lise-exploratuxf3ria-apuxf3s-as-mudanuxe7as}{%
\section{Análise exploratória após as
mudanças}\label{anuxe1lise-exploratuxf3ria-apuxf3s-as-mudanuxe7as}}

\begin{Shaded}
\begin{Highlighting}[]
\FunctionTok{require}\NormalTok{(ggplot2)}
\end{Highlighting}
\end{Shaded}

\begin{verbatim}
## Loading required package: ggplot2
\end{verbatim}

\begin{Shaded}
\begin{Highlighting}[]
\FunctionTok{library}\NormalTok{(gridExtra)}
\end{Highlighting}
\end{Shaded}

\begin{verbatim}
## 
## Attaching package: 'gridExtra'
\end{verbatim}

\begin{verbatim}
## The following object is masked from 'package:dplyr':
## 
##     combine
\end{verbatim}

\begin{Shaded}
\begin{Highlighting}[]
\NormalTok{plot1 }\OtherTok{\textless{}{-}} \FunctionTok{ggplot}\NormalTok{(bal,}\FunctionTok{aes}\NormalTok{(}\AttributeTok{x =}\NormalTok{ os)) }\SpecialCharTok{+} \FunctionTok{geom\_bar}\NormalTok{()}
\NormalTok{plot2 }\OtherTok{\textless{}{-}} \FunctionTok{ggplot}\NormalTok{(bal,}\FunctionTok{aes}\NormalTok{(}\AttributeTok{x =}\NormalTok{ app)) }\SpecialCharTok{+} \FunctionTok{geom\_bar}\NormalTok{()}
\NormalTok{plot3 }\OtherTok{\textless{}{-}} \FunctionTok{ggplot}\NormalTok{(bal, }\FunctionTok{aes}\NormalTok{(}\AttributeTok{x =}\NormalTok{ channel)) }\SpecialCharTok{+} \FunctionTok{geom\_bar}\NormalTok{()}
\NormalTok{plot4 }\OtherTok{\textless{}{-}} \FunctionTok{ggplot}\NormalTok{(bal, }\FunctionTok{aes}\NormalTok{(}\AttributeTok{x =}\NormalTok{ device\_n)) }\SpecialCharTok{+} \FunctionTok{geom\_bar}\NormalTok{()}

\FunctionTok{grid.arrange}\NormalTok{(plot1,plot2,plot3,plot4, }\AttributeTok{ncol =} \DecValTok{2}\NormalTok{)}
\end{Highlighting}
\end{Shaded}

\includegraphics{analise_fraude_files/figure-latex/unnamed-chunk-4-1.pdf}

\hypertarget{algumas-variuxe1veis-preditoras-principalmente-app-e-device_n-estuxe3o-desbalanceadas-poderuxe1-isto-causar-perda-de-precisuxe3o}{%
\section{Algumas variáveis preditoras (principalmente app e device\_n)
estão desbalanceadas, poderá isto causar perda de
precisão?}\label{algumas-variuxe1veis-preditoras-principalmente-app-e-device_n-estuxe3o-desbalanceadas-poderuxe1-isto-causar-perda-de-precisuxe3o}}

\begin{Shaded}
\begin{Highlighting}[]
\FunctionTok{table}\NormalTok{(bal}\SpecialCharTok{$}\NormalTok{app)}
\end{Highlighting}
\end{Shaded}

\begin{verbatim}
## 
##    0-20 100-200   20-40 200-550  40-100 
##   91438     605   10487     146    2337
\end{verbatim}

\hypertarget{criauxe7uxe3o-do-modelo}{%
\section{Criação do modelo}\label{criauxe7uxe3o-do-modelo}}

\begin{Shaded}
\begin{Highlighting}[]
\FunctionTok{library}\NormalTok{(randomForest)}
\end{Highlighting}
\end{Shaded}

\begin{verbatim}
## randomForest 4.6-14
\end{verbatim}

\begin{verbatim}
## Type rfNews() to see new features/changes/bug fixes.
\end{verbatim}

\begin{verbatim}
## 
## Attaching package: 'randomForest'
\end{verbatim}

\begin{verbatim}
## The following object is masked from 'package:gridExtra':
## 
##     combine
\end{verbatim}

\begin{verbatim}
## The following object is masked from 'package:ggplot2':
## 
##     margin
\end{verbatim}

\begin{verbatim}
## The following object is masked from 'package:dplyr':
## 
##     combine
\end{verbatim}

\begin{Shaded}
\begin{Highlighting}[]
\FunctionTok{head}\NormalTok{(bal)}
\end{Highlighting}
\end{Shaded}

\begin{verbatim}
##     app    os channel          click_time is_attributed device_n
## 1  0-20  0-15 400-500 2017-11-07 09:30:38             0        1
## 2 20-40 15-30 200-300 2017-11-07 13:40:27             0        1
## 3  0-20 15-30 200-300 2017-11-07 18:05:24             0        1
## 4  0-20  0-15 400-500 2017-11-07 04:58:08             0        1
## 5  0-20  0-15 100-200 2017-11-09 09:00:09             0        1
## 6  0-20 15-30 100-200 2017-11-09 01:22:13             0        1
\end{verbatim}

\begin{Shaded}
\begin{Highlighting}[]
\FunctionTok{class}\NormalTok{(bal}\SpecialCharTok{$}\NormalTok{is\_attributed)}
\end{Highlighting}
\end{Shaded}

\begin{verbatim}
## [1] "integer"
\end{verbatim}

\begin{Shaded}
\begin{Highlighting}[]
\NormalTok{bal}\SpecialCharTok{$}\NormalTok{is\_attributed }\OtherTok{\textless{}{-}} \FunctionTok{as.factor}\NormalTok{(bal}\SpecialCharTok{$}\NormalTok{is\_attributed)}
\NormalTok{bal}\SpecialCharTok{$}\NormalTok{app }\OtherTok{\textless{}{-}} \FunctionTok{as.factor}\NormalTok{(bal}\SpecialCharTok{$}\NormalTok{app)}
\NormalTok{bal}\SpecialCharTok{$}\NormalTok{os }\OtherTok{\textless{}{-}} \FunctionTok{as.factor}\NormalTok{(bal}\SpecialCharTok{$}\NormalTok{os)}
\NormalTok{bal}\SpecialCharTok{$}\NormalTok{channel }\OtherTok{\textless{}{-}} \FunctionTok{as.factor}\NormalTok{(bal}\SpecialCharTok{$}\NormalTok{channel)}
\NormalTok{bal}\SpecialCharTok{$}\NormalTok{device\_n }\OtherTok{\textless{}{-}} \FunctionTok{as.factor}\NormalTok{(bal}\SpecialCharTok{$}\NormalTok{device\_n)}
\NormalTok{modelo1 }\OtherTok{\textless{}{-}} \FunctionTok{randomForest}\NormalTok{(is\_attributed }\SpecialCharTok{\textasciitilde{}}\NormalTok{ app }\SpecialCharTok{+}\NormalTok{ os }\SpecialCharTok{+}\NormalTok{ channel }\SpecialCharTok{+}\NormalTok{ device\_n,}
                        \AttributeTok{data =}\NormalTok{ bal,}
                        \AttributeTok{mtry =} \DecValTok{2}\NormalTok{,}
                        \AttributeTok{ntree =} \DecValTok{200}\NormalTok{,}
                        \AttributeTok{importance =} \ConstantTok{TRUE}\NormalTok{)}
\NormalTok{modelo1}
\end{Highlighting}
\end{Shaded}

\begin{verbatim}
## 
## Call:
##  randomForest(formula = is_attributed ~ app + os + channel + device_n,      data = bal, mtry = 2, ntree = 200, importance = TRUE) 
##                Type of random forest: classification
##                      Number of trees: 200
## No. of variables tried at each split: 2
## 
##         OOB estimate of  error rate: 2.89%
## Confusion matrix:
##       0    1 class.error
## 0 99186  587 0.005883355
## 1  2452 2788 0.467938931
\end{verbatim}

\begin{Shaded}
\begin{Highlighting}[]
\FunctionTok{varImpPlot}\NormalTok{(modelo1, }\AttributeTok{main =} \StringTok{\textquotesingle{}Nível de importância das variáveis preditoras\textquotesingle{}}\NormalTok{)}
\end{Highlighting}
\end{Shaded}

\includegraphics{analise_fraude_files/figure-latex/unnamed-chunk-6-1.pdf}

\hypertarget{observa-se-que-os-parece-ser-a-uxfanica-variuxe1vel-com-relativo-baixo-nuxedvel-de-importuxe2ncia-frente-as-demais}{%
\section{Observa-se que `os' parece ser a única variável com relativo
baixo nível de importância frente as
demais}\label{observa-se-que-os-parece-ser-a-uxfanica-variuxe1vel-com-relativo-baixo-nuxedvel-de-importuxe2ncia-frente-as-demais}}

\hypertarget{escolhemos-manter-todas-as-variuxe1veis-preditoras}{%
\section{Escolhemos manter todas as variáveis
preditoras}\label{escolhemos-manter-todas-as-variuxe1veis-preditoras}}

\hypertarget{dividindo-em-dados-de-treino-e-de-teste}{%
\section{Dividindo em dados de treino e de
teste}\label{dividindo-em-dados-de-treino-e-de-teste}}

\begin{Shaded}
\begin{Highlighting}[]
\NormalTok{dtreino }\OtherTok{\textless{}{-}}\NormalTok{ bal }\SpecialCharTok{\%\textgreater{}\%} \FunctionTok{slice\_sample}\NormalTok{(}\AttributeTok{prop =} \FloatTok{0.75}\NormalTok{,}\AttributeTok{replace =}\NormalTok{ T)}
\NormalTok{dteste }\OtherTok{\textless{}{-}}\NormalTok{ bal }\SpecialCharTok{\%\textgreater{}\%} \FunctionTok{slice\_sample}\NormalTok{(}\AttributeTok{prop =} \FloatTok{0.25}\NormalTok{,}\AttributeTok{replace =}\NormalTok{ T)}
\FunctionTok{str}\NormalTok{(dteste)}
\end{Highlighting}
\end{Shaded}

\begin{verbatim}
## 'data.frame':    26253 obs. of  6 variables:
##  $ app          : Factor w/ 5 levels "0-20","100-200",..: 1 3 1 1 1 1 1 1 1 1 ...
##  $ os           : Factor w/ 4 levels "0-15","15-30",..: 2 1 2 2 2 2 1 2 1 2 ...
##  $ channel      : Factor w/ 5 levels "0-100","100-200",..: 5 2 3 2 3 2 2 5 3 2 ...
##  $ click_time   : POSIXct, format: "2017-11-08 03:40:33" "2017-11-06 22:38:54" ...
##  $ is_attributed: Factor w/ 2 levels "0","1": 1 1 2 1 1 1 1 1 1 1 ...
##  $ device_n     : Factor w/ 4 levels "0","1","2","Other": 2 2 2 2 2 2 2 2 2 2 ...
\end{verbatim}

\begin{Shaded}
\begin{Highlighting}[]
\FunctionTok{str}\NormalTok{(dtreino)}
\end{Highlighting}
\end{Shaded}

\begin{verbatim}
## 'data.frame':    78759 obs. of  6 variables:
##  $ app          : Factor w/ 5 levels "0-20","100-200",..: 1 5 1 1 1 3 1 1 1 1 ...
##  $ os           : Factor w/ 4 levels "0-15","15-30",..: 1 2 2 1 2 2 1 2 1 1 ...
##  $ channel      : Factor w/ 5 levels "0-100","100-200",..: 2 5 5 5 3 2 3 2 4 5 ...
##  $ click_time   : POSIXct, format: "2017-11-08 06:43:48" "2017-11-09 02:41:37" ...
##  $ is_attributed: Factor w/ 2 levels "0","1": 1 1 1 1 1 1 1 1 1 1 ...
##  $ device_n     : Factor w/ 4 levels "0","1","2","Other": 2 2 2 2 2 3 2 2 2 2 ...
\end{verbatim}

\hypertarget{criauxe7uxe3o-da-segunda-versuxe3o-agora-com-dados-divididos}{%
\section{Criação da segunda versão, agora com dados
divididos}\label{criauxe7uxe3o-da-segunda-versuxe3o-agora-com-dados-divididos}}

\begin{Shaded}
\begin{Highlighting}[]
\NormalTok{modelo2 }\OtherTok{\textless{}{-}} \FunctionTok{randomForest}\NormalTok{(is\_attributed }\SpecialCharTok{\textasciitilde{}}\NormalTok{ app }\SpecialCharTok{+}\NormalTok{ os }\SpecialCharTok{+}\NormalTok{ channel }\SpecialCharTok{+}\NormalTok{ device\_n,}
                        \AttributeTok{data =}\NormalTok{ dtreino,}
                        \AttributeTok{mtry =} \DecValTok{2}\NormalTok{,}
                        \AttributeTok{ntree =} \DecValTok{500}\NormalTok{,}
                        \AttributeTok{importance =} \ConstantTok{TRUE}\NormalTok{)}
\NormalTok{modelo2}
\end{Highlighting}
\end{Shaded}

\begin{verbatim}
## 
## Call:
##  randomForest(formula = is_attributed ~ app + os + channel + device_n,      data = dtreino, mtry = 2, ntree = 500, importance = TRUE) 
##                Type of random forest: classification
##                      Number of trees: 500
## No. of variables tried at each split: 2
## 
##         OOB estimate of  error rate: 3.03%
## Confusion matrix:
##       0    1 class.error
## 0 74146  490 0.006565196
## 1  1896 2227 0.459859326
\end{verbatim}

\hypertarget{retiraremos-os-para-testar-como-fica-a-taxa-de-erro-visto-que-uxe9-a-variuxe1vel-com-menor-nuxedvel-de-importuxe2ncia}{%
\section{Retiraremos `os' para testar como fica a taxa de erro, visto
que é a variável com menor nível de
importância}\label{retiraremos-os-para-testar-como-fica-a-taxa-de-erro-visto-que-uxe9-a-variuxe1vel-com-menor-nuxedvel-de-importuxe2ncia}}

\begin{Shaded}
\begin{Highlighting}[]
\NormalTok{modelo2}\FloatTok{.1} \OtherTok{\textless{}{-}} \FunctionTok{randomForest}\NormalTok{(is\_attributed }\SpecialCharTok{\textasciitilde{}}\NormalTok{ app  }\SpecialCharTok{+}\NormalTok{ channel }\SpecialCharTok{+}\NormalTok{ device\_n,}
                        \AttributeTok{data =}\NormalTok{ dtreino,}
                        \AttributeTok{mtry =} \DecValTok{2}\NormalTok{,}
                        \AttributeTok{ntree =} \DecValTok{500}\NormalTok{)}
\NormalTok{modelo2}\FloatTok{.1}
\end{Highlighting}
\end{Shaded}

\begin{verbatim}
## 
## Call:
##  randomForest(formula = is_attributed ~ app + channel + device_n,      data = dtreino, mtry = 2, ntree = 500) 
##                Type of random forest: classification
##                      Number of trees: 500
## No. of variables tried at each split: 2
## 
##         OOB estimate of  error rate: 3.14%
## Confusion matrix:
##       0    1 class.error
## 0 74090  546 0.007315505
## 1  1926 2197 0.467135581
\end{verbatim}

\hypertarget{a-taxa-de-erro-aumentou-de-fato-retirar-os-nuxe3o-uxe9-uma-boa-alternativa}{%
\section{A taxa de erro aumentou, de fato retirar `os' não é uma boa
alternativa}\label{a-taxa-de-erro-aumentou-de-fato-retirar-os-nuxe3o-uxe9-uma-boa-alternativa}}

\hypertarget{utilizando-o-muxe9todo-naive-bayes}{%
\section{Utilizando o método Naive
Bayes}\label{utilizando-o-muxe9todo-naive-bayes}}

\begin{Shaded}
\begin{Highlighting}[]
\FunctionTok{library}\NormalTok{(e1071)}
\NormalTok{modelo3 }\OtherTok{\textless{}{-}} \FunctionTok{naiveBayes}\NormalTok{(is\_attributed }\SpecialCharTok{\textasciitilde{}}\NormalTok{ app  }\SpecialCharTok{+}\NormalTok{ os }\SpecialCharTok{+}\NormalTok{ channel }\SpecialCharTok{+}\NormalTok{ device\_n, }
                      \AttributeTok{data =}\NormalTok{ dtreino)}
\NormalTok{modelo3}
\end{Highlighting}
\end{Shaded}

\begin{verbatim}
## 
## Naive Bayes Classifier for Discrete Predictors
## 
## Call:
## naiveBayes.default(x = X, y = Y, laplace = laplace)
## 
## A-priori probabilities:
## Y
##          0          1 
## 0.94765043 0.05234957 
## 
## Conditional probabilities:
##    app
## Y           0-20      100-200        20-40      200-550       40-100
##   0 0.8871992068 0.0033227933 0.0925022777 0.0006967147 0.0162790074
##   1 0.5522677662 0.0552995392 0.2418142130 0.0164928450 0.1341256367
## 
##    os
## Y         0-15      15-30      30-45      Other
##   0 0.41222734 0.48485985 0.06661665 0.03629616
##   1 0.32549115 0.56027165 0.05699733 0.05723987
## 
##    channel
## Y        0-100    100-200    200-300    300-400    400-500
##   0 0.01623881 0.32409293 0.31462029 0.10754864 0.23749933
##   1 0.10962891 0.26364298 0.44627698 0.09192336 0.08852777
## 
##    device_n
## Y             0           1           2       Other
##   0 0.004635833 0.943633099 0.044375368 0.007355700
##   1 0.220713073 0.650254669 0.008488964 0.120543294
\end{verbatim}

\hypertarget{fazendo-as-previsuxf5es}{%
\section{Fazendo as previsões}\label{fazendo-as-previsuxf5es}}

\begin{Shaded}
\begin{Highlighting}[]
\NormalTok{predict3 }\OtherTok{\textless{}{-}} \FunctionTok{predict}\NormalTok{(modelo3,}\AttributeTok{newdata =}\NormalTok{ dteste)}
\NormalTok{dteste}\SpecialCharTok{$}\NormalTok{Previsao }\OtherTok{\textless{}{-}}\NormalTok{ predict3}
\FunctionTok{table}\NormalTok{(dteste}\SpecialCharTok{$}\NormalTok{Previsao)}
\end{Highlighting}
\end{Shaded}

\begin{verbatim}
## 
##     0     1 
## 25656   597
\end{verbatim}

\begin{Shaded}
\begin{Highlighting}[]
\FunctionTok{table}\NormalTok{(dteste}\SpecialCharTok{$}\NormalTok{is\_attributed)}
\end{Highlighting}
\end{Shaded}

\begin{verbatim}
## 
##     0     1 
## 24950  1303
\end{verbatim}

\begin{Shaded}
\begin{Highlighting}[]
\NormalTok{cm }\OtherTok{\textless{}{-}} \FunctionTok{table}\NormalTok{(dteste}\SpecialCharTok{$}\NormalTok{is\_attributed,dteste}\SpecialCharTok{$}\NormalTok{Previsao)}


\FunctionTok{library}\NormalTok{(caret)}
\end{Highlighting}
\end{Shaded}

\begin{verbatim}
## Loading required package: lattice
\end{verbatim}

\begin{Shaded}
\begin{Highlighting}[]
\FunctionTok{confusionMatrix}\NormalTok{(cm)}
\end{Highlighting}
\end{Shaded}

\begin{verbatim}
## Confusion Matrix and Statistics
## 
##    
##         0     1
##   0 24752   198
##   1   904   399
##                                           
##                Accuracy : 0.958           
##                  95% CI : (0.9555, 0.9604)
##     No Information Rate : 0.9773          
##     P-Value [Acc > NIR] : 1               
##                                           
##                   Kappa : 0.4013          
##                                           
##  Mcnemar's Test P-Value : <2e-16          
##                                           
##             Sensitivity : 0.9648          
##             Specificity : 0.6683          
##          Pos Pred Value : 0.9921          
##          Neg Pred Value : 0.3062          
##              Prevalence : 0.9773          
##          Detection Rate : 0.9428          
##    Detection Prevalence : 0.9504          
##       Balanced Accuracy : 0.8166          
##                                           
##        'Positive' Class : 0               
## 
\end{verbatim}

\begin{Shaded}
\begin{Highlighting}[]
\NormalTok{predict2 }\OtherTok{\textless{}{-}} \FunctionTok{predict}\NormalTok{(modelo2,}\AttributeTok{newdata =}\NormalTok{ dteste)}
\NormalTok{cm2 }\OtherTok{\textless{}{-}} \FunctionTok{table}\NormalTok{(dteste}\SpecialCharTok{$}\NormalTok{is\_attributed,predict2)}
\FunctionTok{confusionMatrix}\NormalTok{(cm2)}
\end{Highlighting}
\end{Shaded}

\begin{verbatim}
## Confusion Matrix and Statistics
## 
##    predict2
##         0     1
##   0 24785   165
##   1   607   696
##                                           
##                Accuracy : 0.9706          
##                  95% CI : (0.9685, 0.9726)
##     No Information Rate : 0.9672          
##     P-Value [Acc > NIR] : 0.0009217       
##                                           
##                   Kappa : 0.6286          
##                                           
##  Mcnemar's Test P-Value : < 2.2e-16       
##                                           
##             Sensitivity : 0.9761          
##             Specificity : 0.8084          
##          Pos Pred Value : 0.9934          
##          Neg Pred Value : 0.5342          
##              Prevalence : 0.9672          
##          Detection Rate : 0.9441          
##    Detection Prevalence : 0.9504          
##       Balanced Accuracy : 0.8922          
##                                           
##        'Positive' Class : 0               
## 
\end{verbatim}

\hypertarget{o-modelo-randomforest-possui-uma-acuruxe1cia-melhor-assim-como-outros-atributos}{%
\section{O modelo RandomForest possui uma acurácia melhor, assim como
outros
atributos}\label{o-modelo-randomforest-possui-uma-acuruxe1cia-melhor-assim-como-outros-atributos}}

\end{document}
